\newcommand{\R}{\ensuremath{\mathbb{R}}}
\newcommand{\norm}[1]{\left\lVert #1 \right\rVert}
\newcommand{\scalar}[2]{\left<#1,#2\right>}
\renewcommand{\phi}{\varphi}
\renewcommand{\le}{\leqslant}
\renewcommand{\ge}{\geqslant}
\renewcommand{\baselinestretch}{2}
\renewcommand{\epsilon}{\varepsilon}

\newtheorem{theorem}{Теорема}

\documentclass[11pt]{article}

\usepackage{floatrow,graphicx,calc}
\DeclareFloatSeparators{mysep}{\hspace{3cm}}

\usepackage[hidelinks, unicode]{hyperref}
\usepackage{cmap}
\usepackage{a4wide}
\usepackage[utf8]{inputenc}
\usepackage[russian]{babel}
\usepackage{graphicx}
\usepackage{amsmath}
\usepackage{amssymb}
\usepackage{vmargin}
\setpapersize{A4}
\setmarginsrb{2.5cm}{2cm}{1.5cm}{2cm}{0pt}{0mm}{0pt}{13mm}
\usepackage{indentfirst}
\sloppy

\begin{document}

\thispagestyle{empty}

\begin{center}
\ \vspace{-3cm}

\includegraphics[width=0.5\textwidth]{msu.eps}\\
{\scshape Московский государственный университет имени М.~В.~Ломоносова}\\
Факультет вычислительной математики и кибернетики\\
Кафедра системного анализа

\vfill

{\LARGE Отчёт по практикуму}

\vspace{1cm}

{\Huge\bfseries <<Линейная задача ОУ>>}
\end{center}

\vspace{1cm}

\begin{flushright}
  \large
  \textit{Студент 315 группы}\\
  Ф.\,А.~Федоров

  \vspace{5mm}

  \textit{Руководитель практикума}\\
  к.ф.-м.н., доцент П.\,А.~Точилин
\end{flushright}

\vfill

\begin{center}
Москва, 2021
\end{center}

\newpage
\tableofcontents

\newpage
\section{Постановка задачи}


Рассматривается линейная задача быстродействия:
\begin{equation}\label{mainTask}
\begin{aligned}
&\dot x = Ax + Bu, t\in\left[t_0, +\infty    \right],
\\
&x \in \mathbb{R}^2, A \in \mathbb{R}^{2\times2}, B \in \mathbb{R}^{2\times2},\: \forall t, u(t) \in \mathcal{P}
\subset \mathbb{R}^2,
x_0 = x(t_0) \in \chi_0, x_1 = x(t_1) \in \chi_1.
\\
& x(t_1) \in \chi_1, t_1 \rightarrow \inf.
\end{aligned}
\end{equation}
На управляющий $u(t)$ параметр наложено ограничение $u(t) \in \mathcal{P} \: \forall t,$ где $\mathcal{P}$ ~--- невырожденный эллипсоид
с матрицей конфигурации $Q, \chi_0$ - начальное множество, $\chi_1$ - конечное. Необходимо численно решить задачу быстродействия
то есть написать программу, которая принимает на вход параметры, заданные в задаче, а так же параметры, определяющие точность решения, 
численно находит такое управление, за которое траектория системы, выпущенная из начального множества в момент времени
$t_0$ может попасть в некоторую точку целевого множества за минимально допустимое время, 
построить график этой траектории, либо вывести сообщение, что такой траектории нет.

\begin{equation}
\begin{aligned}
&\chi_0 = \{ \left(x_1, x_2\right)' : x_1^2 + (x_2-\alpha)^4 \le \gamma   \},
\\
&\chi_1 = \{\left(x_1, x_2\right)' : x_1x_2 > 1, x_1 \ge 0, x_2 + x_2 < a, a > 0\}.
\end{aligned}
\end{equation}

\section{Опорные функции}
Рассмотрим первое множество: оно выпукло, замкнуто и ограничено $\Rightarrow$ может быть
описано исчерпывающим образом своей опорной функцией.
\[
\rho(l | \chi_0) = \sup_{x \in \chi_0}\scalar{l}{x}.
\] 
Так как множество выпукло $\Rightarrow$ 
максимум скалярного произведения достигается на границе, то есть необходимо найти условный экстремум
\begin{equation}\label{condExtr}
\begin{aligned}
& x_1l_1 + x_2l_2 \rightarrow \sup,
\\
& \left(x_1, x_2\right)' : x_1^2 + (x_2-\alpha)^4 = \gamma.
\end{aligned}
\end{equation}  
Составим функцию Лагранжа:
\[
\mathcal{L} = x_1l_1 + x_2l_2 - \lambda \left(x_1^2 + (x_2-\alpha)^4 - \gamma   \right).
\]
Из равенства частных производных нулю, найдем $\lambda:$
\[
\begin{aligned}
&\frac{\partial\mathcal{L}}{\partial x_1} = l_1 - 2\lambda x_1 = 0 
\Rightarrow x_1 = \frac{l_1}{2\lambda},
\\
&\frac{\partial\mathcal{L}}{\partial x_2} = l_2 - 4\lambda \left(x_2 - \alpha \right)^3 = 0 
\Rightarrow x_2 = \left( \frac{l_2}{4\lambda} \right)^{\frac{1}{3}} + \alpha,
\\
& \frac{\partial\mathcal{L}}{\partial \lambda} = -x_1^2 - (x_2-\alpha)^4 + \gamma = 0.
\\
\end{aligned}
\]
Подставив выражения для $x_1, x_2$ получим
\[
\begin{aligned}
&\left(\frac{l_1}{2\lambda}  \right) ^ 2 + \left(\frac{l_2}{4\lambda}  \right)^{\frac{4}{3}} = \gamma,
\\
& \lambda = t^{-\frac{3}{2}},
\\
& \left(\frac{l_1}{2}   \right)^2 t^3 + \left(\frac{l_2}{4}\right)^{\frac{4}{3}}t^2 = \gamma.
\end{aligned}
\]
Исследуем получившийся многочлен при $l_1 \ne 0$. Пусть
\[
\begin{aligned}
&\left(\frac{l_1}{2}  \right) ^ 2 = a_1, \left(\frac{l_2}{4}  \right)^{\frac{4}{3}} = a_2,
\\
& f(t) = a_1t^3 + a_2t^2 - \gamma, f'(t) = 3a_1t^2 + 2a_2t = t(3a_1t+2a_2).
\end{aligned}
\]
Приравнивая производную нулю, получаем, что
\[
\begin{aligned}
&f(t) \uparrow,  t \in \left(-\infty, -\frac{2a_2}{3a_1}  \right), f(-\frac{2a_2}{3a_1}) = \frac{4a_2^3}{27a_1^2} - \gamma
\\
&f(t) \downarrow , t \in \left(-\frac{2a_2}{3a_1}, 0  \right), f(0) = -\gamma \le 0,
\\
&f(t) \uparrow , t \in \left(0 , \infty \right).
\end{aligned}
\]
Из этого можно сделать вывод, что у многочлена всегда есть 1 вещественный корень больше нуля, и может быть еще два, меньше нуля.
Для выбора подходящего корня, обратимся к \eqref{condExtr}: подставив в выражение для максимизации $x_1, x_2$ выраженные через 
$\lambda$ по описанным выше формулам, получим:
\[
\frac{l_1^2}{2\lambda} + \frac{l_2^{\frac{4}{3}}}{4\lambda ^{\frac{1}{3}}} + \alpha l_2 \rightarrow \sup.
\]
Заметим, что $l_1, l_2$ стоят в четных степенях, $\Rightarrow$ выражение принимает большее значение, при $\lambda > 0, \Rightarrow$
при отборе необходим тот корень, который больше нуля.
Подставляя его, находим $\lambda = t^{-\frac{3}{2}}$, а затем, по описанным выше формулам, находим 
опорный вектор $x = (x_1,x_2)'.$
В случае, когда $l_1 = 0$ имеем квадратное уравнение $a_2 t^2  - \gamma = 0,$ которое очевидно имеет корень больше нуля.

Второе множество представляет из себя пересечение надграфика гиперболы $x_1x_2 \ge 1$ в первом квадранте 
и ‘подграфика’ прямой $x_1+x_2 = a$. Найдем точки пересечения:
\[
\begin{aligned}
&x_2 = a - x_1 = \frac{1}{x_1},
\\
&x_1^2 - ax_1 + 1 = 0,
\\
&D = a^2 - 4
\end{aligned}
\]
Если дискриминант меньше нуля, то множества не пересекаются и $\chi_1 = \varnothing$. Иначе:
\[
\begin{aligned}
&\hat x_1 = \frac{1}{2}\left(a + \sqrt{D} \right),
\\
&\hat x_2 = \frac{1}{2}\left(a - \sqrt{D} \right).
\end{aligned}
\]

Выпишем явно касательные векторы к гиперболе в точках $\hat x_1, \hat x_2:$
обозначим $\phi_1$ ~--- угол касательной к гиперболе в точке $\hat x_1$, $\phi_2$ ~--- в точке $\hat x_2.$
Из свойств производной функции $x_2(x_1) = \frac{1}{x_1}$ следует, что $\tg(\phi_1) = -\frac{1}{\hat x_1^2}.$
Тогда вектор касательной к гиперболе $h_1$ в точке $\hat x_1$ можно вычислить, как вектор, проходящий вдоль прямой $x_2 = x_1 \tg(\phi_1)$
в направлении уменьшения координаты $x_1$ для дальнейшего удобства:
\[
h_1 = \left(-1, -\tg(\phi_1) \right)' = \left(-1, \frac{1}{\hat x_1^2} \right)'.
\]
Используя аналогичные соображения, вычислим вектор касательной к гиперболе $h_2$
в направлении увеличения координаты $x_1$:
\[
h_2 = \left(1, \tg(\phi_2) \right)' = \left(1, -\frac{1}{\hat x_2^2} \right). 
\]

Так как множество $\chi_1$ выпукло $\Rightarrow$ $\sup_{x \in \chi_1} \scalar{l}{x}$ будет достигаться
на том $x,$ для которого $l$ будет перпендикулярно касательной к множеству в точке $x.$
Заметим, что касательные векторы с выбранными направлениями со стороны гиперболы образуют тупой угол, а со стороны прямой ~--- острый.
Проиллюстрировать это может пример множества $\chi_1.$
\newpage

\begin{figure}[h!]
\ffigbox{\caption{Пример $\chi_1$}\label{fig:fbb}}%
{\includegraphics[scale=1]{chi1}}
\end{figure}


То есть можно использовать знаки скалярного произведения, для выявления стороны, к которой вектор $l$ будет нормалью.
Если $\scalar{l}{h_1} < 0$ и $\scalar{l}{h_2} < 0,$ тогда вектор $l$ будет нормалью к гиперболе
то есть необходимо решить задачу минимизации на множестве. 
\[
\begin{aligned}
& x_1l_1 + x_2l_2 \rightarrow \sup,
\\
& \left(x_1, x_2\right)' : x_1 x_2 = 1.
\end{aligned}
\]
Составим функцию Лагранжа:
\[
\mathcal{L} = x_1l_1 + x_2l_2 - \lambda \left(x_1 x_2 - 1 \right).
\]
Из равенства частных производных нулю, найдем $\lambda:$
\[
\begin{aligned}
&\frac{\partial\mathcal{L}}{\partial x_1} = l_1 - \lambda x_2 = 0,
\Rightarrow x_2 = \frac{l_1}{\lambda},
\\
&\frac{\partial\mathcal{L}}{\partial x_2} = l_2 - \lambda x_1 = 0,
\Rightarrow x_1 = \frac{l_2}{\lambda},
\\
& \frac{\partial\mathcal{L}}{\partial \lambda} = -x_1 x_2 + 1 = 0.
\\
\end{aligned}
\]
Подставив выражения для $x_1, x_2$ и учтя, что $x \in I $ четверти, получим
\[
\begin{aligned}
&\frac{l_1}{\lambda} \frac{l_2}{\lambda} = 1 \Rightarrow \lambda = \pm\sqrt{l_1 l_2},
\\
&\left(x_1, x_2\right)' = \left(\sqrt{\frac{l_2}{l_1}},  \sqrt{\frac{l_1}{l_2}}\right)',
\\
&\rho\left(l | \chi_1  \right) = \scalar{x}{l} = 2\sqrt{l_1 l_2}.
\end{aligned}
\]
Заметим, что если вектор $l$ является нормалью к гиперболе, то ни одна его компонента не может быть 
равна нулю так как находится он в III четверти, то есть все рассуждения корректны.

Если $l_1 > 0, l_2 > 0, \frac{l_2}{l_1} = 1,$ тогда один из опорных векторов $x$ коллинеарен
вектору $l$, т. е. $x = \beta l$ и лежащий на прямой $x_1 + x_2 = a \Rightarrow$
\[
\begin{aligned}
& \beta \left(l_1 + l_2  \right) = a \Rightarrow \beta = \frac{a}{l_1+l_2} \Rightarrow 
x = \frac{a}{l_1+l_2} l,
\\
& \rho\left(l | \chi_1  \right) = \scalar{x}{l} = \frac{a}{l_1+l_2}\scalar{l}{l}.
\end{aligned}
\]
Заметим, что в этом случае опорным множеством будет весь отрезок. 

В иных случаях, опорными векторами будут точки пересечения гиперболы и прямой. 
Положим $y_1 = \left(\hat x_1, \frac{1}{\hat x_1} \right)',
y_2 = \left(\hat x_2, \frac{1}{\hat x_2} \right)'$
Тогда если $\scalar{l}{y_2} \le \scalar{l}{y_1}:$
\[
\begin{aligned}
&x = y_1,
\\
& \rho\left(l | \chi_1  \right) = \scalar{y_1}{l},
\end{aligned}
\]
иначе
\[
\begin{aligned}
& x = y_2,
\\
& \rho\left(l | \chi_1  \right) = \scalar{y_2}{l}.
\end{aligned}
\]

\section{Принцип максимума}
Для решения задачи численно, потребуется принцип максимума в формулировке для линейной задачи быстродействия.
\begin{theorem}\label{PMP}

Пусть допустимое управление $u(t) \in \mathcal{P}, t \in \left[t_0, t_1    \right]$ 
переводящее фазовую точку из положения $x_0 = x(t_0) \in \chi_0$ в положение $x_1 \in \chi_1,$
а $x(t)$ ~--- соответствующая траектория, то есть решение системы из \eqref{mainTask} при заданном $u(t),$
такая, что $x(t_0) = x_0, x(t_1) = x_1.$ Для оптимальности по быстродействию управления $u(t)$
и траектории $x(t)$ необходимо существование такой ненулевой непрерывной вектор-функции
$\psi(t) = (\psi_1(t), \psi_2(t), \dots, \psi_n(t))':$
\begin{enumerate}
\item $ \dot \psi(t) = -A'(t)\psi(t), \psi(t) \ne 0,$
\item $\scalar{\psi(t)}{B(t)u(t)} = \rho(\psi(t)| B(t) \mathcal{P}\left(t\right)), $
\item $\scalar{\psi(t_0)}{x(t_0)} = \rho(\psi(t_0) | \chi_0),$
\item $\scalar{-\psi (t_1)}{x(t_1)} = \rho(-\psi(t_1) | \chi_1).$
\end{enumerate}
\end{theorem}

\section{Решение задачи}
Из \eqref{PMP} и множества управлений получим $(u(\psi)).$ Для этого воспользуемся известной формулой для
опорной функции эллипса: 
\[
\rho(l | \mathcal{P}) = \sqrt{\scalar{Ql}{l}} + \scalar{l}{p}, x = \frac{Ql}{\sqrt{\scalar{Ql}{l}}} + p.
\]
Нетрудно заметить, что 
\begin{multline*}
\scalar{\psi(t)}{ B(t)u(t)} = \{u(t) = \hat u(t) + p, \hat u(t \in \mathcal{E}(Q, 0))   \} = 
\scalar{\psi(t)}{ B(t)\hat u(t)} + \scalar{\psi(t)}{B(t)p} = \rho \left(\psi(t) | B(t)\mathcal{P}\right) = 
\\
\rho \left(\psi(t) | B(t)\mathcal{E}(Q, 0) \right) + \scalar{\psi(t)}{B(t)p},
\end{multline*}
где $\mathcal{E}(Q, 0) $ ~--- эллипс с центром в нуле и матрицей конфигурации $Q.$ Сокращая из обеих частей равенства 
$\scalar{\psi(t)}{B(t)p}$ получим, что можно для простоты выразить управление для эллипса с центом в нуле, а затем добавить к нему 
вектор смещения эллипса $p.$

Так как матрица $Q = Q'$ и $Q > 0 \Rightarrow \exists R > 0 : Q = RR'.$
Далее, множество 
\[
\mathcal{P} = \{x: \scalar{Q^{-1}x}{x} \le 1  \} = R\tilde{B}_1(0), \tilde{B}_1(0) = \{x : \scalar{x}{x} \le 1 \}.
\] 
Перепишем условие 2 из \eqref{PMP}, подставив конкретные значения и опустив аргумент для краткости:
\[
\begin{aligned}
& \scalar{\psi}{Bu} = \scalar{B'\psi}{u} = \{u \in \mathcal{P} = R\tilde{B}(0, 1) \Rightarrow
\exists v \in \tilde{B}_1(0) : Rv = u\} = \scalar{B'\psi}{Rv} =
\\ 
&\scalar{R'B'\psi}{v} \le
\norm{R'B'\psi}_2\norm{v}_2 = \sqrt{\scalar{QB'\psi}{B'\psi}} = \rho(B'\psi |\mathcal{P}) = \rho(\psi | B \mathcal{P}).
\end{aligned}
\]
Цепочка данных преобразований справедлива, когда выполнено неравенство КБ для почти всюду $t$, то есть когда 
\[
v = \beta R'B'\psi \Rightarrow 
v = \frac{R'B'\psi}{\norm{R'B'\psi}_2} \Rightarrow u = Rv = \frac{QB'\psi}{\norm{R'B'\psi}_2} = \frac{QB'\psi}{\sqrt{\scalar{QB'\psi}{B'\psi}}}.
\]
Таким образом было найдено явное выражение для $u(\psi(t))$ при \( \sqrt{\scalar{QB'\psi}{B'\psi}} \ne 0 .\)
В случае равенства нулю, из положительной определенности матрицы $Q$ следует, что 
\[ \sqrt{\scalar{QB'\psi}{B'\psi}} = 0 \Rightarrow B'\psi = 0.\]
То есть \(u\) может быть выбран любым. Положим в данном случае
\(u\) равным своему предыдущему ненулевому значению.

Зная конкретное $u(t),$ можем найти $x(t),$ решая численно уравнения для $x(t)$ из \eqref{mainTask} при заданном управлении.
\section{Алгоритм решения}
\begin{enumerate}
\item Проверка множеств $\chi_0, \chi_1$ на вырожденность.
\item Перебираем все направления для \(\psi_0\) вида 
\(  \left(\cos\left(\frac{\pi k}{N}\right) , \sin\left(\frac{\pi k}{N}  \right)   \right)  .\) От величины $N$ будет
зависеть точность перебора, то есть при его увеличении точность численного решения увеличивается.
\item Для каждого направления \(\psi_0\) численно решаем задачу Коши (с помощью функции ode45) для $\psi.$
В силу ограниченности возможностей компьютера, $\psi(t)$ будет кусочно постоянным (пусть и с маленькими промежутками времени)
и будет считаться до заранее заданного $T.$
\item По траектории $\psi$ восстанавливаем кусочно постоянное управление (оно будет кусочно постоянным в силу замечания, сделанного в предыдущем пункте), 
а, зная управление, решаем задачу Коши для $x(t)$ (с помощью функции ode45).
\item Выбираем наилучшее решение. Если траекторий, пересекающих итоговое множество нет, то завершаем работу.
\item Если оптимальная траектория найдена, тогда выделяем эту траекторию черным цветом, выводим графики $u(t), \psi(t), x(t)$, и,
 пользуясь вторым условием трансверсальности, считаем погрешность по формуле
$| \scalar{\psi(t_1)}{x(t_1)} - \rho(-\psi(t_1) | \chi_1) |,$ где $t_1 $ ~--- конечное время, а вектор $\psi(t_1) $ ~--- нормирован.

\section{Переменные программы}
\(a, \alpha, \gamma, A, B, Q, p\) ~--- переменные, значения которых понятны из постановки задачи.
\(N\) ~--- число направлений, вдоль которых решается уравнение для $\psi.$
$T$ ~--- максимально допустимое время интегрирования траектории.
$\lambda_1, \lambda_2$ ~--- собственные значения матрицы $A.$
На выходе программа выдает график траекторий, построенных в процессе поиска оптимальной, и, 
если оптимальная найдена, то выделяет среди них оптимальную черной линией, выводит погрешность
решения, график компонент для оптимального управления, оптимальной траектории и сопряженной переменной.

\end{enumerate}


\section{Примеры}
Ниже приведены изображения нескольких видов. Комментарии к изображениям (номер комментария соответствует виду изображения):
\begin{enumerate}
\item На данных изображениях представлены траектории, полученные в процессе перебора значений для $\psi_0.$
Зеленые линии ~--- траектории, полученные в ходе перебора начальных значений $\psi_0$, 
черная линия ~--- оптимальная траектория, красные линии ~--- границы множества $\chi_0,$ 
синие ~--- границы множества $\chi_1.$
\item На данных изображениях представлены графики компонент численно подобранного ‘оптимального’ управления
в зависимости от времени.
\item На данных изображениях представлены графики компонент численно подобранной ‘оптимальной’ траектории
в зависимости от времени.
\item На данных изображениях представлены графики компонент численно подобранной ‘оптимальной’ сопряженной
переменной $\psi$ в зависимости от времени.
\end{enumerate}

\newpage
\[a = 10, \alpha = 0, \gamma = 0.1, \lambda_1 = 0, \lambda_2 = 1, p = (1,0)',\] 
\[T_{max} = 1, T_{opt} = 0.3142, N = 100, \delta = 9.1\cdot 10^{-6},\]
\[
A = \left(
\begin{array}{cc}
0 & 0\\
0 & 1\\
\end{array}
\right),
B = \left(
\begin{array}{cc}
1 & 0\\
0 & 1\\
\end{array}
\right),
Q = \left(
\begin{array}{cc}
2 & 0\\
0 & 4\\
\end{array}
\right)
\]

\begin{figure}[h!]
\ffigbox{\caption{График вида 1}\label{fig:fbb}}%
{\includegraphics[scale=1]{test1}}
\end{figure}

\begin{figure}\caption{График вида 2}
\centering
\includegraphics[scale=0.6]{u1}
\end{figure}

\begin{figure}\caption{График вида 3}
\centering
\includegraphics[scale=0.6]{x1}
\end{figure}

\begin{figure}\caption{График вида 4}
\centering
\includegraphics[scale=0.6]{psi1}
\end{figure}

\newpage
\[a = 5, \alpha = 1, \gamma = 0.01, \lambda_1 = 0.783, \lambda_2 = 10.217, p = (1,0)',\] 
\[T_{max} = 0.1, T_{opt} = 0.0585, N = 30, \delta = 3.9\cdot 10^{-6},\]
\[
A = \left(
\begin{array}{cc}
1 & 2\\
1 & 10\\
\end{array}
\right),
B = \left(
\begin{array}{cc}
1 & 0\\
0 & 1\\
\end{array}
\right),
Q = \left(
\begin{array}{cc}
2 & 0\\
0 & 4\\
\end{array}
\right)
\]

\begin{figure}[h!]
\ffigbox{\caption{График вида 1}\label{fig:fbb}}%
{\includegraphics[scale=1]{test2}}
\end{figure}

\begin{figure}\caption{График вида 2}
\centering
\includegraphics[scale=0.6]{u2}
\end{figure}

\begin{figure}\caption{График вида 3}
\centering
\includegraphics[scale=0.6]{x2}
\end{figure}

\begin{figure}\caption{График вида 4}
\centering
\includegraphics[scale=0.6]{psi2}
\end{figure}
\newpage

\newpage
Следующие 2 примера иллюстрируют тот факт, что задача не является непрперывно зависящей от начального множества $\chi_0:$
при расширении начального множества на $0.0001$ оптимальное время изменяется с $14$ до $0.$
\[a = 20, \alpha = 8, \gamma = 0.01, \lambda_1 = \frac{1}{10}i\sqrt{3}, \lambda_2 = -\frac{1}{10}i\sqrt{3}, p = (1,0)',\] 
\[T_{max} = 18, T_{opt} = 14.5431, N = 300, \delta = 6.7\cdot 10^{-5},\]
\[
A = \left(
\begin{array}{cc}
0 & -\frac{2}{10}\\
\frac{3}{10} & 0\\
\end{array}
\right),
B = \left(
\begin{array}{cc}
\frac{1}{5} & 0\\
0 & \frac{1}{5}\\
\end{array}
\right),
Q = \left(
\begin{array}{cc}
2 & 1\\
1 & 4\\
\end{array}
\right)
\]

\begin{figure}[h!]
\ffigbox{\caption{График вида 1}\label{fig:fbb}}%
{\includegraphics[scale=1]{test3}}
\end{figure}

\begin{figure}\caption{График вида 2}
\centering
\includegraphics[scale=0.6]{u3}
\end{figure}

\begin{figure}\caption{График вида 3}
\centering
\includegraphics[scale=0.6]{x3}
\end{figure}

\begin{figure}\caption{График вида 4}
\centering
\includegraphics[scale=0.6]{psi3}
\end{figure}
\newpage

\newpage
\[a = 20, \alpha = 8, \gamma = 0.02, \lambda_1 = \frac{1}{10}i\sqrt{3}, \lambda_2 = -\frac{1}{10}i\sqrt{3}, p = (1,0)',\] 
\[T_{max} = 18, T_{opt} = 0, N = 300, \delta = 0,\]
\[
A = \left(
\begin{array}{cc}
0 & -\frac{2}{10}\\
\frac{3}{10} & 0\\
\end{array}
\right),
B = \left(
\begin{array}{cc}
\frac{1}{5} & 0\\
0 & \frac{1}{5}\\
\end{array}
\right),
Q = \left(
\begin{array}{cc}
2 & 1\\
1 & 4\\
\end{array}
\right)
\]

\begin{figure}[h!]
\ffigbox{\caption{График вида 1}\label{fig:fbb}}%
{\includegraphics[scale=1]{test4}}
\end{figure}
\newpage

\[a = 5, \alpha = 0, \gamma = 0.1, \lambda_1 = 1, \lambda_2 = -5, p = (1,0)',\] 
\[T_{max} = 1, T_{opt} = 0.6877, N = 100, \delta = 2.7\cdot 10^{-4},\]
\[
A = \left(
\begin{array}{cc}
1 & 0\\
0 & -5\\
\end{array}
\right),
B = \left(
\begin{array}{cc}
1 & 0\\
0 & 1\\
\end{array}
\right),
Q = \left(
\begin{array}{cc}
2 & 1\\
1 & 4\\
\end{array}
\right)
\]

\begin{figure}[h!]
\ffigbox{\caption{График вида 1}\label{fig:fbb}}%
{\includegraphics[scale=1]{test5}}
\end{figure}

\begin{figure}\caption{График вида 2}
\centering
\includegraphics[scale=0.6]{u5}
\end{figure}

\begin{figure}\caption{График вида 3}
\centering
\includegraphics[scale=0.6]{x5}
\end{figure}

\begin{figure}\caption{График вида 4}
\centering
\includegraphics[scale=0.6]{psi5}
\end{figure}
\newpage

На примере, представленном ниже видно, как после изменение итогового множества на $0.0001$ 
оптимальное время изменилось на 4, что говорит об отсутствии непрерывной зависимости.

\[a = 5.2499, \alpha = -15, \gamma = 1,  p = (1,0)',\] 
\[T_{max} = 11, T_{opt} = 10.8, N = 50,\]
\[
A = \left(
\begin{array}{cc}
\sin(t) & -2\\
3 & \cos(t)\\
\end{array}
\right),
B = \left(
\begin{array}{cc}
-1 & 0\\
0 & -1\\
\end{array}
\right),
Q = \left(
\begin{array}{cc}
2 & 1\\
1 & 4\\
\end{array}
\right)
\]

\begin{figure}[h!]
\ffigbox{\caption{График вида 1}\label{fig:fbb}}%
{\includegraphics[scale=1]{test6}}
\end{figure}

\begin{figure}\caption{График вида 2}
\centering
\includegraphics[scale=0.6]{u6}
\end{figure}

\begin{figure}\caption{График вида 3}
\centering
\includegraphics[scale=0.6]{x6}
\end{figure}

\begin{figure}\caption{График вида 4}
\centering
\includegraphics[scale=0.6]{psi6}
\end{figure}
\newpage

\[a = 5.25, \alpha = -15, \gamma = 1,  p = (1,0)',\] 
\[T_{max} = 11, T_{opt} = 6.1, N = 50,\]
\[
A = \left(
\begin{array}{cc}
\sin(t) & -2\\
3 & \cos(t)\\
\end{array}
\right),
B = \left(
\begin{array}{cc}
-1 & 0\\
0 & -1\\
\end{array}
\right),
Q = \left(
\begin{array}{cc}
2 & 1\\
1 & 4\\
\end{array}
\right)
\]

\begin{figure}[h!]
\ffigbox{\caption{График вида 1}\label{fig:fbb}}%
{\includegraphics[scale=1]{test7}}
\end{figure}

\begin{figure}\caption{График вида 2}
\centering
\includegraphics[scale=0.6]{u7}
\end{figure}

\begin{figure}\caption{График вида 3}
\centering
\includegraphics[scale=0.6]{x7}
\end{figure}

\begin{figure}\caption{График вида 4}
\centering
\includegraphics[scale=0.6]{psi7}
\end{figure}
\newpage




\newpage
\begin{thebibliography}{9}
\bibitem{lektures} \emph{Лекции по курсу оптимальное управление}.
\bibitem{mhogoznak} Арутюнов А. В. \emph{Лекции по выпуклому и многозначному анализу}. ФИЗМАТЛИТ, 2014.
\bibitem{OC} Л.С. Понтрягин, В.Г. Болтянский, Р.В. Гамкрелидзе, Е.Ф. Мищенко \emph{Математическая теория оптимальных процессов}.
МОСКВА ‘НАУКА’, ГЛАВНАЯ РЕДАКЦИЯ ФИЗИКО-МАТЕМАТИЧЕСКОЙ ЛИТЕРАТУРЫ, 1983.


\end{thebibliography}
\end{document}

















