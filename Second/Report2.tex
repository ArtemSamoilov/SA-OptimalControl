\newcommand{\R}{\ensuremath{\mathbb{R}}}
\newcommand{\norm}[1]{\left\lVert #1 \right\rVert}
\newcommand{\scalar}[2]{\left<#1,#2\right>}
\renewcommand{\phi}{\varphi}
\renewcommand{\le}{\leqslant}
\renewcommand{\ge}{\geqslant}
\renewcommand{\baselinestretch}{2}
\renewcommand{\epsilon}{\varepsilon}
\newcommand{\hham}{{\mathscr{H}}}
\newcommand{\mmam}{{\mathscr{M}}}
\newcommand{\xxam}{{\mathscr{X}}}



\newtheorem{theorem}{Теорема}
\newtheorem{propos}{Утверждение}

\newtheorem{proof}{Доказательство}
\newtheorem{comment}{Замечание}

%\documentclass[11pt]{article}
\documentclass[10pt,pdf,hyperref={unicode}]{beamer}

\usepackage{floatrow,graphicx,calc}
\DeclareFloatSeparators{mysep}{\hspace{3cm}}


\usepackage[hidelinks, unicode]{hyperref}
\usepackage{cmap}
\usepackage{a4wide}
\usepackage[utf8]{inputenc}
\usepackage[russian]{babel}
\usepackage{graphicx}
\usepackage{amsmath}
\usepackage{amssymb}
\usepackage{vmargin}
\usepackage{mathrsfs}
\setpapersize{A4}
\setmarginsrb{2.5cm}{2cm}{1.5cm}{2cm}{0pt}{0mm}{0pt}{13mm}
\usepackage{indentfirst}
\sloppy

\DeclareMathOperator*{\argmax}{argmax}
\DeclareMathOperator*{\argmin}{argmin}

\begin{document}
	
	\thispagestyle{empty}
	
	\begin{center}
		\ \vspace{-3cm}
		
		{\scshape Московский государственный\\
		Ордена Ленина, Ордена Октябрьской Революции,\\
		Ордена Красного Трудового Знамени\\
		университет имени М.~В.~Ломоносова}\\
		Факультет вычислительной математики и кибернетики\\
		Кафедра системного анализа
		
		%\vfill
		\vspace{1cm}
		\begin{center}
		{\includegraphics[width=6cm]{mgu}}
		\end{center}
		\vspace{1cm}
		
		{\LARGE Отчёт по практикуму на ЭВМ}
		
		\vspace{1cm}
		
		{\Huge\bfseries <<Управление ракетой>>}
		\end{center}
		
		\vspace{1cm}
		
		\begin{flushright}
		  \large
		  \textit{Студент 315 группы}\\
		  А.\,А.~Самойлов
		
		  \vspace{5mm}
		
		  \textit{Руководитель практикума}\\
		  к.ф.-м.н., доцент П.\,А.~Точилин
		\end{flushright}
		
		\vspace{5cm}
		
		\begin{center}
		  2021
		\end{center}
		
		\newpage
	
	\tableofcontents
	
	\newpage
	\section{Постановка задачи}
	Движение ракеты в вертикальной плоскости над поверхностью планеты описывается дифференциальными уравнениями:
	
	\begin{equation}\label{rawSystem}
	\left\{
		\begin{aligned}
		\dot m v + m \dot v &= -gm - kv^2 + lu, \\
		\dot m &= -u.\\
		\end{aligned}
	\right.
	\end{equation}
	Здесь \(v \in \R \) ~--- скорость ракеты, \(m \) ~--- ее переменная масса, \(g > 0\) ~--- гравитационная постоянная,
	\(k > 0\) ~--- коэффициент трения \(l > 0\) ~--- коэффициент, определяющий силу, действующую на ракету со стороны
	сгорающего топлива, \(u \in \left[u_{min}, u_{max}  \right]\) ~--- скорость подачи топлива \(\left(u_{max} > u_{min} > 0 \right)  .\)
	Кроме того, известна масса корпуса ракеты без топлива \(M > 0.\)
	
	\textbf{Задача 1.} Задан начальный момент времени \(t_0 > 0, \) начальная скорость
	\(v(0) = 0, \) а также начальная масса ракеты с топливом \(m(0) = m_0 > M.\) Необходимо за счет выбора программного
	управления \(u(t)\) перевести ракету на максимально возможную высоту в заданное временя \(T > 0.\)
	
	\textbf{Задача 2.} Задан начальный момент времени \(t_0 > 0, \) начальная скорость
	\(v(0) = 0, \) а также начальная масса ракеты с топливом \(m(0) = m_0 > M.\) Необходимо за счет выбора программного
	управления \(u(t)\) перевести ракету на заданную высоту \(H > 0\) в заданный момент времени
	\(T > 0\) так, чтобы при этом минимизировать функционал 
	\[
	J = \int \limits ^T_0 e^{-\alpha t} u(t)dt + \beta v(T), \quad \alpha > 0, \beta \ge 0.
	\]
	
	
	Необходимо написать в среде MatLab программу, которая по заданным параметрам
	\(T, M, m_0, u_{min}, u_{max}, l, k, g, H, \alpha\) определит, разрешима ли задача
	оптимального управления. Если задача разрешима, то программа должна построить график
	компонент оптимального управления, компонент оптимальной траектории, сопряженных
	переменных. Кроме того, программа должна определить количество переключений найденного оптимального управления, 
	а также моменты переключения.
	
	\newpage
	
	\section{Формализация}
	Введем обобщенные координаты: 
	\(x_1 = \dot x\) ~--- скорость ракеты,
	\(x_2 = m\) ~--- масса ракеты.
	\(x_3 = x\) ~--- высота, на которую поднялась ракета, 
	В этих обозначениях, после вычитания из первого уравнения второго, домноженного на $v$, 
	и деление обоих частей первого уравнения на $m$, 
	добавления граничных условий, множества ограничений на управление, система \eqref{rawSystem} примет вид:
	
	\begin{equation}\label{mainSystem}
	\left\{
		\begin{aligned}
		&\dot x_1 = -g - \frac{k x_1^2}{x_2} + \frac{lu}{x_2} + \frac{u x_1}{x_2},\\
		&\dot x_2 = -u,\\
		&\dot x_3 = x_1, \\
		&x_1(0) = 0, \\
		&x_2(0) = m_0,\\
		&x_3(0) = 0,\\
		&u \in [u_{min}, u_{max}].\\
		\end{aligned}
	\right.
	\end{equation}
	
	
	
	\section{Необходимые теоретические сведения}
	В дальнейшем, нам потребуется
	\begin{theorem}[Частный случай ПМП]
	Пусть задана задача оптимального управления, вида
	\begin{equation}\label{mainSystem}
	\left\{
		\begin{aligned}
		& \dot x = f(x, u),  t \in \left[t_0, t_1\right]\\
		& x(t_0) = x_0,\\
		& x_0 \rightarrow \xxam ^ 1, \\
		& u \in U, \\
		& J = \int \limits ^{t_1}_{t_0} f_0(u) du \rightarrow \inf ,\\
		& x(t) \in \R ^n \;\forall t \in [t_0, t_1], \quad f = (f_1, f_2, \dots , f_n) ^\prime .\\
		\end{aligned}
	\right.
	\end{equation}
	
	Введем обозначения 
	\[
	\tilde x = (x_0, x_1, \dots, x_n) ^\prime, \tilde f = (f_0, f_1, \dots , f_n) ^\prime,
	\]
	
	\( \tilde \hham (\tilde \psi, \tilde x, u)  = \scalar{\tilde f}{\tilde \psi}, \tilde \psi \in \R^{n+1}\) ~--- функция Гамильтона-Понтрягина.
	
	Пусть \(  \{x^*, u^*\}  \) ~--- оптимальная пара. 
	
	Тогда \(\exists \tilde \psi ^ * : [t_0, t_1] \rightarrow \R^{n+1}\) такая, что:
	\begin{enumerate}
	\item \(\tilde \psi ^ * \not\equiv 0   \)  для п.в. \(t \in [t_0, t_1];\)
	\item \( \tilde \psi ^ * = -\frac{\partial \tilde \hham}{\partial \tilde x} \left(\tilde \psi ^ *,  \tilde x^*, u^*  \right)  \) 
		для п.в. \(t \in [t_0, t_1];\)
	\item \(  \tilde \hham \left(\tilde \psi ^ *,  \tilde x^*, u^*  \right) = 
		\sup \limits_{u \in U} \tilde \hham \left(\tilde \psi ^ *,  \tilde x^*, u  \right) = 
		\tilde \mmam \left( \tilde \psi ^ *,  \tilde x^* \right)\) для п.в. \(t \in [t_0, t_1];\)  
	\item
		\[
		\begin{aligned}
		&\tilde \psi^*_0 (.) = const \le 0,\\
		&\tilde \mmam \left( \tilde \psi ^ *,  \tilde x^* \right) \equiv const = 0,\\
		\end{aligned}
		\] 
	\item \(\psi^*(t_1) \perp T_{x^*(t_1)} \xxam^1,\)  ~--- касательная гиперплоскость к множеству $\xxam^1$ в точке $x^*(t_1).$
	\end{enumerate}
	\end{theorem}

	\newpage
	\section{Задача 1}
	\subsection{Построение сопряженной системы}
	В задаче ракета начинает движение из положения $x_1 = 0, x_2 = m_0, x_3 = 0.$
	Из постановки задачи ясно, что множество итоговое множество $\xxam^1$ будет иметь вид 
	\(\xxam^1 = \{ (x_1, x_2) ^ \prime \in \R^2 \mid x_2 \ge M\},\)
	\begin{equation}\label{Task1}
	\left\{
	\begin{aligned}
	&\dot x_1 = -g - \frac{k x_1^2}{x_2} + \frac{lu}{x_2} + \frac{u x_1}{x_2},\\
	&\dot x_2 = -u,\\
	&x_1(0) = 0, x_2(0) = m_0,\\
	& x(0) \rightarrow x(T) \in \xxam^1 = \{ (x_1, x_2) ^ \prime \in \R^2 \mid x_2 \ge M\},\\
	&u \in [u_{min}, u_{max}].\\
	\end{aligned}
	\right.
	\end{equation}

	Для применения ПМП необходимо как-то выделить момент времени, когда заканчивается топливо и динамика системы меняется. 
	Будем вести перебор по времени $T_1 \in [0, T]$ ~--- момент окончания топлива.
	C этого момента динамика системы меняется, и она начинает описываться следующим уравнением:
	\begin{equation}\label{remSystem}
	\left\{
	\begin{aligned}
	& \dot x_1 = -g - \frac{kx_1^2}{M},\\
	& x_2 \equiv M,\\
	& x_1(T_1) = v_1,\\
	\end{aligned}
	\right.
	\end{equation}
	где $v_1 = x_1(T_1)$ ~--- скорость, набранная к моменту времени $T_1.$ 
	Как будет показано далее, если траектория оптимальная, то $x_1(t) > 0$ на $[0, T_1].$
	$x_1(t)$ в системе \eqref{remSystem} явно выражается в виде
	\[
	x_1(t) = R_1 \tg (R_2(C_1M - t)),
	\]
	где $R_1 = \sqrt{\frac{gM}{k}}, R_2 = \sqrt{\frac{gk}{M}}, C_1$ ~--- подбирается из граничных условий.
	Чтобы не вести перебор по скорости, будем вести перебор по времени, когда скорость стала нулевой. 
	Пусть $T_2 \ge T_1: \dot x_1(T_2) = 0.$ Далее будет показано, что либо $T_2 > T$ и этот случай анализируется отдельно, либо
	Максимальная высота достигается в момент времени $T_2.$ Добавим к перебираемым переменным $T_2.$
	Зная $T_2,$ вычислим $С_1, x_1(T_1):$ 
	\[
	\begin{aligned}
	x_1(T_2) = 0 = R_1\tg R_2(C_1M - T_2) \Rightarrow C_1 = \frac{T_2}{M},\\
	x_1(T_1) = R_1\tg R_2(T_2 - T_1) = v_1.
	\end{aligned}
	\]

	Будем применть ПМП на отрезке $(0, T_1),$ с граничным условием $x_1(T_1) = v_1.$
	То есть конечное время будет фиксировано. Известо, что для того, чтобы свести случай с 
	фиксированным временем, к классическому ПМП с нефиксированным конечным моментом, 
	придется вводить новые переменные $x_4(t) = t, \psi_4(t),$ при этом, используя условие (4) ПМП (условие максимума), 
	$\psi_4$ можно однозначно выразить на любом момента времени. Таким образом, можно упростить общую систему, 
	убрав эти переменные, но отказавшись от использования условия максимума.

	Введем новую координату, отвечающую за минимизируемый фукнционал:
	\[
	\begin{aligned}
	& x_0(t) = -\int \limits_0^{t} x_1(\tau)d\tau.\\
	\end{aligned}
	\]
	и новые переменные
	\[
	\begin{aligned}
	& \tilde x(t) = (x_0(t), x_1(t), x_2(t)) ^ \prime, \\
	& f_0(x, u) = -x_1(t), \\
	& \tilde f = (f_0, f)^\prime.\\
	\end{aligned}
	\]
	Составим функцию Гамильтона-Понтрягина:
	\[
	\tilde \hham(\tilde \psi, \tilde x, u) = \scalar{\tilde f}{\tilde \psi} = 
	-\psi_0 x_1 + 
	\psi_1\left(-g - \frac{k x_1^2}{x_2} + \frac{lu}{x_2} + \frac{u x_1}{x_2}   \right) - 
	\psi_2 u.
	\]
	Следуя условию (3) ПМП, вычислим максимизатор $\tilde \hham$ по $u.$ 
	Преобразуем выражением к следующему виду:
	\[
	\tilde \hham(\tilde \psi, \tilde x, u) = 
	-\psi_0 x_1 - 
	\psi_1\left(g + \frac{kx_1^2}{x_2}   \right) + 
	u\left(\psi_1\frac{l + x_1}{x_2} - \psi_2  \right) \rightarrow \sup \limits_{u \in U}.
	\]

	Учитывая, что $x_2 > 0$, видно, что максимизатор будет равен:
	\begin{equation*}
	u^*(\tilde \psi, \tilde x, t) = 
	\begin{cases}
	u_{max}, & G(t) > 0,\\
	\left[u_{min}, u_{max}\right] &G(t) = 0,\\
	u_{min}, & G(t) < 0.\\
	\end{cases}
	\end{equation*}
	Следуя условию (2) ПМП, составим систему для $\tilde \psi$ и объединим ее с уравнениями из \eqref{Task1}
	\begin{equation}\label{System1}
	\left\{
	\begin{aligned}
	& \dot x_0 = -x_1,\\
	& \dot x_1 = -g - \frac{k x_1^2 - u\left(l + x_1\right)}{x_2},\\
	& \dot x_2 = -u,\\
	& x_0(0) = 0, x_1(0) = 0, x_2(0) = m_0,\\
	& x_1(T_1) = v_1, x_2(T_1) = M,\\
	& \dot \psi_0 = 0, \\
	& \dot \psi_1 = \psi_0 + \psi_1\frac{2k x_1 - u}{x_2}, \\
	& \dot \psi_2 = \psi_1\frac{-kx_1^2 + lu + ux_1}{x_2^2},\\
	& \psi_0(0) = \psi_0^0, \psi_1(0) = \psi_1^0, \psi_2(0) = \psi_2^0,\\
	& u \in [u_{min}, u_{max}].\\
	\end{aligned}
	\right.
	\end{equation}
	Так как итоговое множество - точка $(x_1, x_2) = (v_1, 0) \Rightarrow$ условие (5) ПМП не может дать никакой информации.

	\subsection{Анализ системы}
	Обозначим 
	\[
	F(x_1) = -kx_1^2 + ux_1 + lu - gx_2.
	\]
	Видно, что это парабола, ветви которой направлены вниз, т. к. $k < 0.$
	Найдем особые точки для координаты $x_1:$
	\[
	\begin{aligned}
	\dot x_1 = 0 \Rightarrow -kx_1^2 + ux_1 + lu - gx_2 = 0,\\
	D[u] = u^2 + 4k(-gx_2 + lu);
	\end{aligned}
	\]
	Если $D < 0,$ то $F(x_1) < 0$ и ракета в этом случае теряет скорость.
	\[
	\begin{aligned}
	&\hat x_1 &= \frac{u - \sqrt{D}}{2k},\\
	&\hat x_2 &= \frac{u + \sqrt{D}}{2k}.\\
	\end{aligned}
	\]

	Из ориентации параболы следует, что $F(x) < 0, x > \hat x_2.$
	\[
	\begin{aligned}
	& F(x) < 0, x < \hat x_1,\\
	& F(x) > 0, \hat x_1 < x < \hat x_2,\\
	& F(x) < 0, \hat x_2 < x.
	\end{aligned}
	\]
	Тогда при $x_1 > \hat x_2 \quad \dot x_1 < 0,$ а при $x_1 < \hat x_2 \dot x_1 > 0,$ то есть 
	$\hat x^2$ ~--- устойчивое положение равновесия, $x_1(t) \to \hat x^2.$ 
	Также координата положения равновесия зависит от времени:
	видно, что при фиксированном $u, D[u] \uparrow,$ т. к. $x_2 \downarrow.$
	Это значит, что $\hat x^2 \uparrow,$ парабола $F(x_1)$ поднимается.

	Докажем вспомогательные утверждения:
	\begin{propos}\label{mainProp}
	Если есть две ракеты, описываемые системой \eqref{rawSystem}, $m_1 < m_2$ ~--- массы ракет без топлива,
	при этом управления совпадают, то первая ракета взлетит выше.
	\end{propos}
	\begin{proof}
	Положим $\tilde x_i$ ~--- i-ая координата для первой ракеты, $\hat x_i$ ~--- для второй.
	$\dot {\tilde x}_1$ как функция от $\tilde x_1$ является параболой. При этом, она расположена строго выше, чем
	$\dot {\hat x}_1.$ А так как $\tilde x_1(t) = \int \limits^{t}_{0} \dot {\tilde x}_1(\tau) d\tau,$ то получаем, что
	$\tilde x_1(t) - \hat x_1(t) > 0, $ то есть скорость первой ракеты строго больше старой, от сюда следует, что первая взлетит выше.
	\end{proof}
	\begin{comment}
	Заметим, что если при этом у первой ракеты будет больше начальная скорость или высота,
	то утверждение остается верным.
	\end{comment}

	\begin{propos}
	$x_1(t)$ ~--- непрерывна.
	\end{propos}
	\begin{proof}
	Докажем непрерывность по определению. Пусть $\tau_1, \tau_2: \|\tau_1 - \tau_2\| < \delta.$ 
	Для этого воспользуемся тем, что $x_2 \ge M,$ и т. к. $\hat x_2$ ~--- неподвижная точка для 
	$x_1$, описанная выше, то $x_1 \le \hat x_2 < C = const$ т. к. она выражается через $u, x_2, k,$
	которые ограничены. Не ограничивая общности, будем считать, что $\tau_1 < \tau_2.$
	\[
	\begin{aligned}
	\|  x_1(\tau_1) - x_1(\tau_2) | = 
	& \left| \int_{\tau_1}^{\tau_2} - g - \frac{kx_1^2(t)}{x_2(t)} + \frac{lu}{x_2(t)} + \frac{ux_1(t)}{x_2(t)} dt\right| < \\ 
	& < \int_{\tau_1}^{\tau_2}  g + \frac{kC^2}{M} + \frac{lu_{max}}{M} + \frac{uC}{M} dt = \int_{\tau_1}^{\tau_2} A dt = 
	(\tau_2 - \tau_1)A < \epsilon, \delta < \frac{\epsilon}{A}.\\
	\end{aligned}
	\]
	Здесь $A$ ~--- константное подынтегральное выражение. 

	\end{proof}

	\begin{propos}
		Если для на отрезке $[\tau_0, \tau_1] \quad x_1(t) < 0, $ 
		то либо максимальная высота достигается в момент времени $t_1 < \tau_0,$ либо это управление не оптимально.
	   \end{propos}
	   \begin{proof}
		Подобная ситуация возможна в двух случаях:
		\begin{enumerate}
		 \item Начиная с момента времени $\tau_e < \tau_0$ закончилось топливо. 
		  Рассмотрим момент времени $\tau_e,$ когда $x_2(t) = M,$ т. е. когда закончилось топливо.
		  Обозначим $\tilde x_i = x_i(\tau_e), \tilde \psi_i = \psi_i(\tau_e), i = \overline{0, 2}$
		  На этом отрезке система выглядит так:
	   
		\begin{equation}\label{endFuel}
		   \left\{
			\begin{aligned}
			 &\dot x_0 = -x_1,\\
			 &\dot x_1 = -g - \frac{k x_1^2}{M},\\
			 &\dot x_2 = 0,\\
			 &x_0(\tau_e) = \tilde x_0, x_1(\tau_e) = \tilde x_1 = 0, x_2(\tau_3) = \tilde x_2,\\
			 &\dot \psi_0 = 0, \\
			 &\dot \psi_1 = \psi_0 + \frac{2k x_1}{x_2}, \\
			 & \dot \psi_2 = \frac{-\psi_1kx_1^2}{x_2^2},\\
			 & \psi_0(\tau_e) = \tilde \psi_0, \psi_1(\tau_e) = \tilde \psi_1, \psi_2(\tau_e) = \tilde \psi_2 ,\\
		\end{aligned}
	   \right.
	   \end{equation}

	Видно, что скорость только убывает. Тогда, для максимизации высоты необходимо продолжать 
	интегрирование до тех пор, пока $x_1(t) > 0.$ Как только $x_1(t) = 0, $ дальше она продолжит только уменьшаться,
	и высота также будет уменьшаться.
	\item Произошло переключение с $u_{max}, $ на $u_{min},$ и этого управление не хватает, чтобы 
	увеличивать скорость ракеты. Тогда, возможны два случая:
	\begin{itemize}
		\item $x_1(t) < 0, \forall t > \tau_0.$ В этом случае переключение не имело смысла производить, так как если
		бы ракета осталась на $u_{max},$ то итоговая высота была бы больше. Значит, текущее управление не оптимальное.
		\item $x_1(t) > 0, t > \tau_1.$ В силу непрерывности $x_1$, в какой-то момент времени функция должна была пройти через 0.
		Пусть $x_1(\tau_1) = 0.$ Заменим управление на $u_{max}.$ Обозначим $\tilde x_i$ ~--- параметры при новом управлении.
		В силу того, что $\tilde x_2(t) = m_0 - \int \limits^t_0 u(t) dt$ в какой-то момент времени $\tau \le \tau_2, \tilde x_2(\tau) = x_2(\tau_2),$ 
		но при этом $\tilde x_1(\tau) > 0 > x_1(\tau).$ От сюда можно получить, что ракета при новом управлении взлетит выше, значит
		старое было не оптимальным.  
	\end{itemize} 
	\end{enumerate}
	\end{proof}
	Минимизируемый функционал $J(t_1, x_1) = -\int \limits_0^{t_1} x_1(t)dt$ непрерывен по $t_1, $ а т. к. 
	$x_1(t)$ непрерывен, то получаем, что $J(t_1, x_1) \in C^1[0, T]$ по $t_1.$
	Из необходимого условия экстремума и непрерывности $x_1(t)$ получаем, 
	что минимум достигается, когда $x_1(t) = 0,$ или на граничной точке, то есть когда $t_1 = T.$
	Из достаточного получим, что экстремум достигается, когда $x_1(t)$ меняет знак, то есть
	если $\tau$ ~--- точка, подозрительная на экстремум, то $x_1(\tau + 0) < 0, x_1(\tau - 0) > 0.$
	Таким образом, максимум высоты будет достигаться в тех точках, в которых $x_1$ меняет знак.
	Также выше было показано, что если управление оптимальное, то максимум высоты достигается до того момента, когда
	$x_1(t) < 0.$ То есть для решения задачи достаточно интегрировать до того момента, пока $x_1(t) > 0.$
	Как только $x_1(t) = 0,$ то либо максимум высоты достигается в этой точке при этом управлении, либо управление не оптимальное.

	\begin{propos}
		Оптимальное управление либо будет начинаться в режиме $u_{max},$ либо не будет задействовать этот режим.
	   \end{propos}
	   \begin{proof}
		Предположим, что $u$ ~--- оптимальное управление, которое начинается не с $u(t) = u_{max},$
		и содержит отрезок времени $u(t) \equiv u_{max}, t \in[\tau_0, \tau_1]:$
		\[
		 u(t) = 
		 \left\{
		  \begin{aligned}
		   & w(t) < u_{max} & t \in [0, \tau_0],\\
		   & u_{max}, & t \in [\tau_0, \tau_1].\\
		  \end{aligned}
		 \right.
		\]
		Положим новое управление, а также $\tau = \tau_1 - \tau_0.$
		\[
		 \tilde u(t) = 
		 \left\{
		  \begin{aligned}
		   & u_{max}, & t \in [0, \tau],\\
		   & w(t) < u_{max} & t \in [\tau, \tau_1 - \tau],\\
		  \end{aligned}
		 \right.
		\]
		То есть поменяем режимы местами. Получим, что на момент времени $\tau$
		для ракеты под новым управлением справедливы следующие утверждения: ее масса меньше, скорость больше и набранная высота больше.
		То есть выполнены условия утверждения \eqref{mainProp} 
	   \end{proof}

	\subsection{Решение задачи}
		Рассмотрим ракету в начальный момент времени: из вышеописанных рассуждений следует, 
		что если $\hat x_1 |_{t = 0} > 0,$ то $\dot x_1 < 0,$ значит $x_1 \downarrow.$ Но т. к. 
		$x(t) \ge 0$ по условию, значит ракета не взлетит. Такое возможно, когда $u > \sqrt{D},$
		или $lu > gx_2,$ то есть когда двигатель не может поднять ракету. Для того, чтобы ракета 
		взлетела, необходимо сжечь лишнее горючее. Очевидно, что если ракета это время будет управляться
		в режиме $u(t) = u_{max},$ то это будет оптимальной стратегией, т. к. будет затрачено меньше всего времени.
		Пусть до момента времени $\tau_0$ ракета жгла горючее. Заметим, что во время этого процесса ни одна из переменных, кроме $x_2$
		не поменяла своего значения. То есть можно перейти к новой системе, 
		где $T_{new} = T - \tau_0, m_{new} = m_0 - u_{max}\tau_0.$
		Тогда, для дальнейших теоретических обоснования, не ограничивая общности, будем считать, что в момент времени
		$\dot x_1(0) > 0.$

		Как описывалось ранее, для решения задачи будем вести перебор по $T_1$ ~--- время, когда
		закончилось топливо. До этого момента времени ракета описывается системой \eqref{System1}. 
		Выше было показано, что если управление оптимальное, то
		либо $x_2(T_1) = M,$ либо можно пролететь всю траекторию на управлении $u_{max}.$
		Таким образом, перебор можно ограничить, исходя из следующих соображений:
		\begin{enumerate}
		\item Если для $T_1 = \tilde T$ можно до конца управлять ракетой в режиме $u \equiv u_{max},$
		то для любого меньшего $T_1$ можно сделать тоже самое. Таким образом, перебор надо начинать с момента
		времени $T_0$ ~--- время, которое ракета может пролететь в режиме $u(t) \equiv u_{max}.$
		Заметим еще, что в случае $T_1 = T_0$ будет проведена проверка анормального режима.
		При этом, если $T_0 > T,$ то это управление оптимальное, и задача решена.
		\item Если ракета не может к моменту времени $T_1$ долететь в режиме $u \equiv u_{min},$ чтобы топливо не закончилось,
		то для любого $\tilde T > T_1$ это сделать нельзя. Тогда можно ограничить перебор с верху моментом времени 
		$T_3$ ~--- время, когда заканчивается топливо при управлении в режиме $u \equiv u_{min}.$
		При этом, если $T_3 < 0,$ тогда ракета не взлетит, и ответ к задаче $0.$
		\end{enumerate}

	\subsection{Не хватает топлива}
		Рассмотрим случай, когда ракета не может долететь до момента $T_3$ на $u \equiv u_{max}$ так, чтобы $x_2(T_3) \ge M.$
		Тогда, из ограничений на $T_1$ этого можно добиться с помощью оптимального управления $u^*(t) \not \equiv u_{max}.$

		Пусть ракета с момента времени $\tau_0$ до момента времени $\tau_1$ двигалась под управлением $u(t) = u_{max}.$
		В момент времени $\tau_1$ произошло переключение, то есть стало выполняться равенство
		\begin{equation}\label{psi2}
		\psi_1(\tau_1)(l + x_1(\tau_1)) = \psi_2(\tau_1) x_2(\tau_1) \Leftrightarrow 
		\psi_2(\tau_1) = \frac{\psi_1(\tau_1)\left(l + x_1(\tau_1)\right)}{x_2(\tau_1)}.
		\end{equation}
		Можно выразить $\psi_2(\tau_1)$ через $\psi_1(\tau_1).$
		\subsection{Особый режим}
		Предположим, что в момент $\tau_1$ ракета вошла в особый режим и провела в нем какое-то время.
		Тогда верно, что
		\[
		 G(t) = \psi_1(l + x_1)-\psi_2x_2 = 0.
		\]
		Выразим $u_s$ ~--- управление в особом режиме, продифференцировав выражение для $G(t)$ и приравняв соответствующие производные к значениям из системы \eqref{System1}.
		
		\[
		 \dot G(t) = \psi_0(l + x_1) + \frac{\psi_1}{x_2}2kx_1(l + x_1) - \psi_1 g + u\left(\psi_2 - \frac{\psi_1}{x_2}(l + x_1)\right).
		\]
		Скобка перед $u$ равна нулю, т. к. она равна $\frac{G(t)}{x_2}.$
		Т. к. выражение для второй производной слишком громоздкие, приведем выраженное $u_s$ из уравнение 
		$\ddot G(t) = 0.$
		\begin{equation}\label{specialU}
		 u_s = \frac{2g\psi_0 {x_2}^2 + 2gkl \psi_1 x_2 + 6gk\psi_1 x_1 x_2 - k\psi_0 {x_1}^2 x_2 - 2k^2l\psi_1{x_1}^2 - 2\psi_0kl x_1 x_2}
				{gx_2\psi_1 + 2kl^2\psi_1 + 6klx_1\psi_1 + 4kx_1^2\psi_1 + l\psi_0x_2 + \psi_0 x_1x_2}
		\end{equation}
		Итого, получилось выразить $\psi_2(\tau_1), u_s$ ~--- через $\psi_1(\tau_1), \psi_0(\tau_1)$ которые остается только перебрать.
		
	\subsection{Алгоритм решения задачи}
		Подытожив вышесказанное, соберем все в пошаговый алгоритм:
		\begin{enumerate}
		\item До тех пор, пока ракета не может взлететь, с помощью управления $u(t) = u_{max}$ 
		сжигаем лишнее топливо. Если взлететь до момента $t = T$ не получилось, то ответ задачи - ноль.
		\item Предположим, что в момент времени $\tau_0 < T$ ракета может взлететь. Вычислим $T_0$ ~--- время, которое ракета может
		пролететь на $u_{min}, T_3$ ~--- время, которое ракета может пролететь на $u_{max}.$ Если $T_3 \ge T,$
		значит управление $u \equiv u_{max}$ оптимальное и задача решена.
		\item Перебираем $T_0 \le T_1 \le T_2 \le T_3 \le T, T_1$ ~--- время, когда закончилось топливо, или $x_2(T_1) = M.$
		$T_2$ ~--- время, когда скорость стала равна нулю, или $x_2(T_2) = 0.$ 
		\item Решаем задачу оптимального управления \eqref{System1} на отрезке $[0, T_1].$ 
		Для этого перебираем время первого переключения $\tau_1$ 
		\item В момент времени $\tau_1$ произошло переключение, управление $u_s$ численно выражается по формуле \eqref{specialU},
		$\psi_2(\tau_1)$ по формуле \eqref{psi2}. Значения $(\psi_0(\tau_1), \psi_1(\tau_1)) : (\psi_0(\tau_1)) ^ 2 + (\psi_1(\tau_1)) ^ 2 = 1.$
		будем перебирать. Теперь, систему можно проинтегрировать до конца однозначно.
		\item При переборе выбираем максимальную высоту, которой удалось достичь.
		\end{enumerate}
\end{document}